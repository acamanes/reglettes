\documentclass{article}

\usepackage[utf8]{inputenc}
\usepackage[T1]{fontenc} 
\usepackage[frenchb]{babel}
\usepackage[a4paper, landscape, margin=0.5cm]{geometry}
\usepackage{amssymb}
\usepackage{tikz}
\usetikzlibrary{calc}
\newcommand*\circled[1]{\tikz[baseline=(char.base)]{
        \node[shape=circle,draw,minimum size=5mm, inner sep=1pt,fill=white] (char)
        {\rule[-0.5pt]{-2.6pt}{\dimexpr2ex+0pt}#1};}}


\newcommand{\trianglet}[2]
{\begin{scope}[xshift=#1,yshift=#2,scale=1.1]\draw[fill=black] (-1/4,0) -- (1/4,0) -- (0, {sqrt(3)/4}) -- cycle;\end{scope}}

\newcommand{\carret}[2]
{\begin{scope}[xshift=#1,yshift=#2,scale=1.1]\draw[rotate=0,fill=black] (-1/4,0) -- (1/4,0) -- (1/4, 1/2) -- (-1/4,1/2) -- cycle;\end{scope}}


\begin{document}


\vfill

\begin{center}
  {\Huge\bf
    Nombre d'échanges par parcours de ligne
    }

\vfill
    
\begin{tikzpicture}[scale=0.9]

\draw (1.5,16) node{{\LARGE\bf Cartes}};
\draw (30,16) node{{\LARGE\bf Total}};
  
% Construction des separateurs verticaux
\draw[thick] (0,-1) -- (0,17);
\foreach \i in {2,...,16} {
\draw[thick] (2*\i-1,-1) -- (2*\i-1,17);
};

% Construction des separateurs horizontaux
\foreach \i in {0,...,9} {
\draw[thick] (0,2*\i-1) -- (31,2*\i-1);
};

% Construction des carres et fleches gauches
\foreach \i in {3,5,7,9,11,13} {
\carret{4+2*\i cm}{15.75cm};
};

% Construction des triangles et fleches droites
\foreach \i in {2,4,6,8,10,12,14} {
\trianglet{4+2*\i cm}{15.75 cm};
};
\end{tikzpicture}
\end{center}
\end{document}
