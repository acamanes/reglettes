\documentclass{article}

\usepackage[utf8]{inputenc}
\usepackage[T1]{fontenc} 
\usepackage[frenchb]{babel}
\usepackage[a4paper, landscape, margin=0.5cm]{geometry}
\usepackage{amssymb}
\usepackage{tikz}
\usetikzlibrary{calc}
\newcommand*\circled[1]{\tikz[baseline=(char.base)]{
        \node[shape=circle,draw,minimum size=5mm, inner sep=1pt,fill=white] (char)
        {\rule[-0.5pt]{-2.6pt}{\dimexpr2ex+0pt}#1};}}


\newcommand{\trianglet}[2]
{\begin{scope}[xshift=#1,yshift=#2,scale=1.1]\draw[fill=black] (-1/4,0) -- (1/4,0) -- (0, {sqrt(3)/4}) -- cycle;\end{scope}}

\newcommand{\carret}[2]
{\begin{scope}[xshift=#1,yshift=#2,scale=1.1]\draw[rotate=0,fill=black] (-1/4,0) -- (1/4,0) -- (1/4, 1/2) -- (-1/4,1/2) -- cycle;\end{scope}}


\begin{document}


\vfill


\begin{center}
\begin{tikzpicture}[scale=1.1]
% Construction des references
\draw[fill=black] (-3,15) rectangle (-2,16);
\draw[fill=black] (20,15) rectangle (21,18);

% Construction des separateurs verticaux
\foreach \i in {0,...,10} {
\draw[thick, dashed] (2*\i-1,0) -- (2*\i-1,15);
};

% Construction des carres et fleches gauches
\foreach \i in {1,3,5,7} {
\carret{2*\i cm}{0cm};
\draw [very thick,<->] (2*\i+0.4,0.25) to[bend right] node[above]{\circled{\huge\bf ?}} (2*\i+2-0.4,0.25)
;
};

% Construction des triangles et fleches droites
\foreach \i in {0,2,4,6,8} {
\trianglet{2*\i cm}{1 cm};
\draw [very thick, <->] (2*\i+0.4,1.25) to[bend right] node[above]{\circled{\huge\bf ?}} (2*\i+2-0.4,1.25);
};
\end{tikzpicture}
\end{center}

\newpage

\begin{center}
{\Huge \bf Algorithme}
\end{center}
{\Huge
\begin{enumerate}
\item On parcourt les réglettes de gauche à droite.

Chaque réglette posée dans une colonne $\blacktriangle$ se compare avec sa voisine de droite ; si elle est plus petite, elle reste dans sa colonne, sinon elles échangent leurs places.

\item On parcourt les réglettes de gauche à droite.

Chaque réglette posée dans une colonne $\blacksquare$ se compare avec sa voisine de droite ; si elle est plus petite, elle reste dans sa colonne, sinon elles échangent leurs places.

\item Si, au cours des deux étapes précédentes, aucune réglette n'a changé de place, on recommence les deux étapes précédentes.

Sinon, on s'arrête.
\end{enumerate}
}
\end{document}
